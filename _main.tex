% Options for packages loaded elsewhere
\PassOptionsToPackage{unicode}{hyperref}
\PassOptionsToPackage{hyphens}{url}
%
\documentclass[
]{book}
\usepackage{amsmath,amssymb}
\usepackage{lmodern}
\usepackage{iftex}
\ifPDFTeX
  \usepackage[T1]{fontenc}
  \usepackage[utf8]{inputenc}
  \usepackage{textcomp} % provide euro and other symbols
\else % if luatex or xetex
  \usepackage{unicode-math}
  \defaultfontfeatures{Scale=MatchLowercase}
  \defaultfontfeatures[\rmfamily]{Ligatures=TeX,Scale=1}
\fi
% Use upquote if available, for straight quotes in verbatim environments
\IfFileExists{upquote.sty}{\usepackage{upquote}}{}
\IfFileExists{microtype.sty}{% use microtype if available
  \usepackage[]{microtype}
  \UseMicrotypeSet[protrusion]{basicmath} % disable protrusion for tt fonts
}{}
\makeatletter
\@ifundefined{KOMAClassName}{% if non-KOMA class
  \IfFileExists{parskip.sty}{%
    \usepackage{parskip}
  }{% else
    \setlength{\parindent}{0pt}
    \setlength{\parskip}{6pt plus 2pt minus 1pt}}
}{% if KOMA class
  \KOMAoptions{parskip=half}}
\makeatother
\usepackage{xcolor}
\IfFileExists{xurl.sty}{\usepackage{xurl}}{} % add URL line breaks if available
\IfFileExists{bookmark.sty}{\usepackage{bookmark}}{\usepackage{hyperref}}
\hypersetup{
  pdftitle={Računalniški praktikum (fizika) - vaje},
  pdfauthor={Rok Kuk},
  hidelinks,
  pdfcreator={LaTeX via pandoc}}
\urlstyle{same} % disable monospaced font for URLs
\usepackage{color}
\usepackage{fancyvrb}
\newcommand{\VerbBar}{|}
\newcommand{\VERB}{\Verb[commandchars=\\\{\}]}
\DefineVerbatimEnvironment{Highlighting}{Verbatim}{commandchars=\\\{\}}
% Add ',fontsize=\small' for more characters per line
\usepackage{framed}
\definecolor{shadecolor}{RGB}{248,248,248}
\newenvironment{Shaded}{\begin{snugshade}}{\end{snugshade}}
\newcommand{\AlertTok}[1]{\textcolor[rgb]{0.94,0.16,0.16}{#1}}
\newcommand{\AnnotationTok}[1]{\textcolor[rgb]{0.56,0.35,0.01}{\textbf{\textit{#1}}}}
\newcommand{\AttributeTok}[1]{\textcolor[rgb]{0.77,0.63,0.00}{#1}}
\newcommand{\BaseNTok}[1]{\textcolor[rgb]{0.00,0.00,0.81}{#1}}
\newcommand{\BuiltInTok}[1]{#1}
\newcommand{\CharTok}[1]{\textcolor[rgb]{0.31,0.60,0.02}{#1}}
\newcommand{\CommentTok}[1]{\textcolor[rgb]{0.56,0.35,0.01}{\textit{#1}}}
\newcommand{\CommentVarTok}[1]{\textcolor[rgb]{0.56,0.35,0.01}{\textbf{\textit{#1}}}}
\newcommand{\ConstantTok}[1]{\textcolor[rgb]{0.00,0.00,0.00}{#1}}
\newcommand{\ControlFlowTok}[1]{\textcolor[rgb]{0.13,0.29,0.53}{\textbf{#1}}}
\newcommand{\DataTypeTok}[1]{\textcolor[rgb]{0.13,0.29,0.53}{#1}}
\newcommand{\DecValTok}[1]{\textcolor[rgb]{0.00,0.00,0.81}{#1}}
\newcommand{\DocumentationTok}[1]{\textcolor[rgb]{0.56,0.35,0.01}{\textbf{\textit{#1}}}}
\newcommand{\ErrorTok}[1]{\textcolor[rgb]{0.64,0.00,0.00}{\textbf{#1}}}
\newcommand{\ExtensionTok}[1]{#1}
\newcommand{\FloatTok}[1]{\textcolor[rgb]{0.00,0.00,0.81}{#1}}
\newcommand{\FunctionTok}[1]{\textcolor[rgb]{0.00,0.00,0.00}{#1}}
\newcommand{\ImportTok}[1]{#1}
\newcommand{\InformationTok}[1]{\textcolor[rgb]{0.56,0.35,0.01}{\textbf{\textit{#1}}}}
\newcommand{\KeywordTok}[1]{\textcolor[rgb]{0.13,0.29,0.53}{\textbf{#1}}}
\newcommand{\NormalTok}[1]{#1}
\newcommand{\OperatorTok}[1]{\textcolor[rgb]{0.81,0.36,0.00}{\textbf{#1}}}
\newcommand{\OtherTok}[1]{\textcolor[rgb]{0.56,0.35,0.01}{#1}}
\newcommand{\PreprocessorTok}[1]{\textcolor[rgb]{0.56,0.35,0.01}{\textit{#1}}}
\newcommand{\RegionMarkerTok}[1]{#1}
\newcommand{\SpecialCharTok}[1]{\textcolor[rgb]{0.00,0.00,0.00}{#1}}
\newcommand{\SpecialStringTok}[1]{\textcolor[rgb]{0.31,0.60,0.02}{#1}}
\newcommand{\StringTok}[1]{\textcolor[rgb]{0.31,0.60,0.02}{#1}}
\newcommand{\VariableTok}[1]{\textcolor[rgb]{0.00,0.00,0.00}{#1}}
\newcommand{\VerbatimStringTok}[1]{\textcolor[rgb]{0.31,0.60,0.02}{#1}}
\newcommand{\WarningTok}[1]{\textcolor[rgb]{0.56,0.35,0.01}{\textbf{\textit{#1}}}}
\usepackage{longtable,booktabs,array}
\usepackage{calc} % for calculating minipage widths
% Correct order of tables after \paragraph or \subparagraph
\usepackage{etoolbox}
\makeatletter
\patchcmd\longtable{\par}{\if@noskipsec\mbox{}\fi\par}{}{}
\makeatother
% Allow footnotes in longtable head/foot
\IfFileExists{footnotehyper.sty}{\usepackage{footnotehyper}}{\usepackage{footnote}}
\makesavenoteenv{longtable}
\usepackage{graphicx}
\makeatletter
\def\maxwidth{\ifdim\Gin@nat@width>\linewidth\linewidth\else\Gin@nat@width\fi}
\def\maxheight{\ifdim\Gin@nat@height>\textheight\textheight\else\Gin@nat@height\fi}
\makeatother
% Scale images if necessary, so that they will not overflow the page
% margins by default, and it is still possible to overwrite the defaults
% using explicit options in \includegraphics[width, height, ...]{}
\setkeys{Gin}{width=\maxwidth,height=\maxheight,keepaspectratio}
% Set default figure placement to htbp
\makeatletter
\def\fps@figure{htbp}
\makeatother
\setlength{\emergencystretch}{3em} % prevent overfull lines
\providecommand{\tightlist}{%
  \setlength{\itemsep}{0pt}\setlength{\parskip}{0pt}}
\setcounter{secnumdepth}{5}
\usepackage{booktabs}
\ifLuaTeX
  \usepackage{selnolig}  % disable illegal ligatures
\fi
\usepackage[]{natbib}
\bibliographystyle{plainnat}

\title{Računalniški praktikum (fizika) - vaje}
\author{Rok Kuk}
\date{2022-01-06}

\begin{document}
\maketitle

{
\setcounter{tocdepth}{1}
\tableofcontents
}
\hypertarget{o-strani}{%
\chapter*{O strani}\label{o-strani}}
\addcontentsline{toc}{chapter}{O strani}

Na tej strani so zbrani zapiski za vaje predmeta računalniški praktikum v 1. letniku študija fizike na Fakulteti za matematiko in fiziko Univerze v Ljubljani.

Zapiski so mišljeni le kot opora pri izvajanju vaj in ne obsegajo čisto vseh obravnavanih vsebin. Zapiski zato ne morejo nadomestiti obiskovanja predavanj in vaj.

Ta spletna stran je dostopna na \url{https://python.rokuk.org} in
\url{https://rokuk.github.io/rp-fiz-notes}

Markdown koda za strani je dostopna na \url{https://github.com/rokuk/rp-fiz-notes}
po licenci \href{https://github.com/rokuk/rp-fiz-notes/blob/main/LICENSE}{GNU GPLv2}.

\hypertarget{namestitev-okolja-za-vaje}{%
\chapter{Namestitev okolja za vaje}\label{namestitev-okolja-za-vaje}}

Stran je v delu.

Da na računalniku uporabljate Python in rešujete naloge je potrebno namestiti
nekaj programov. Spodaj je opisan okvirni postopek in pogoste težave pri nameščanju
programov. Če imate težave, je opis problema dobro pogooglati, sicer pa pišite
asistentu ali postavite vprašanje na forumu.

\hypertarget{namestitev-pythona}{%
\section{Namestitev Pythona}\label{namestitev-pythona}}

\begin{enumerate}
\def\labelenumi{\arabic{enumi}.}
\tightlist
\item
  Namestite Python (najbolje kar verzijo 3.10) s te strani (zavihek \texttt{Downloads}): \url{https://www.python.org}
  Če uporabljate Windows 7 ali še starejši Windows, boste morali namestiti
  starejšo verzijo Pythona (npr. 3.8.9 ali manj). Najdete jo tule: \url{https://www.python.org/downloads/}
\end{enumerate}

\begin{quote}
Ko poženete program za namestitev, v oknu, ki se odpre, odkljukajte
``Add Python 3.x to PATH''.
Nato nadaljujte z namestitvijo (opcija Install now).
\end{quote}

\begin{enumerate}
\def\labelenumi{\arabic{enumi}.}
\setcounter{enumi}{1}
\tightlist
\item
  Preverite ali se je Python uspešno namestil. Odprite program Ukazni poziv (Windows)
  ali Terminal (macOS in Linux), ki je že na vašem računalniku. V okno, ki se odpre
  vpišite ukaz \texttt{python\ -\/-version} in pritisnite tipko Enter. Če je Python
  uspešno nameščen, bi se vam v novi vrstici moralo izpisati \texttt{Python\ 3.x.y}
  (kjer je \texttt{x.y} verzija nameščenega Pythona).
\end{enumerate}

\begin{quote}
Če ste na Windowsu in ukaz \texttt{pyhon\ -\/-version} ne izpiše verzije, temveč javi napako, poskusite ukaz \texttt{py\ -\/-version}.
\end{quote}

\hypertarget{namestitev-visual-studio-code}{%
\section{Namestitev Visual Studio Code}\label{namestitev-visual-studio-code}}

\begin{enumerate}
\def\labelenumi{\arabic{enumi}.}
\setcounter{enumi}{2}
\tightlist
\item
  Namestite Visual Studio Code: \url{https://code.visualstudio.com}
\item
  Namestite Python extension - Odprite Visual Studio Code. Morda se vam bo V
  VSCode okonu pojavil zavihek z naslovom \texttt{Get\ Started}, ki ga lahko kar zaprete.
  Na levi kliknite na Extensions (ikona s štirimi kvadratki), vpišite \texttt{Python},
  izberite \texttt{Python} in na desni kliknite \texttt{Install}. Morda se bo odprlo okno
  \texttt{Get\ Started}, ki ga lahko zaprete.
\item
  Dobro je, da si nekje na računalniku ustvarite mapo, v katero boste shranjevali
  vso vašo kodo. V VSCode v meniju kliknite \texttt{Open\ Folder...} in izberite to mapo.
  Morda se bo pojavilo okno, ki vas sprašuje, če zaupate avtorju datotek
  v tej mapi: kliknite Yes, ker sebi zaupate. Ustvarite novo datoteko
  (meni File \textgreater{} New File), jo shranite (meni File \textgreater{} Save) in jo poimenujte
  \texttt{test.py} (na Windowsu izberete \texttt{Save\ as\ type:\ Python}). Vsebina datoteke se vam
  odpre kot zavihek v VSCode. Vanj vpišite \texttt{print("Pozdravljen\ svet!")}. V desnem
  zgornjem kotu bi morali imeti gumb (\texttt{\textbar{}\textgreater{}}), s katerim lahko poženete napisani program.
  Če ga nimate, si lahko namestite extension z imenom \texttt{Code\ Runner}. Sicer lahko
  program poženete tudi tako, da desno kliknete v območju urejevalnika in nato v
  meniju, ki se pojavi izberete \texttt{Run\ Python\ File\ in\ Terminal}.
\item
  Priporočam, da vklopite tudi ``linter''. To je program, ki je del VSCode in
  v vaši kodi sproti preverja ali ste se kje zmotili. Ne ujame vseh možnih napak,
  mnoge pa zazna in vas nanje opozori, še preden poženete program.
  VSCode pritisnite Ctrl+Shift+P in vpišite ``linter'', kliknite na
  \texttt{Python:\ Select\ Linter}. Pojavi se meni z različnimi možnostmi za različne linterje.
  Priporočam \texttt{flake8}, ki nas poleg napak opozori tudi na kršitve priporočil za stil
  PEP8. VSCode vas bo levo spodaj obvestil, da ta linter ni nameščen;
  namestite ga tako, da kliknete \texttt{Install} v tem obvestilu. Po nekaj sekundah bo nameščen.
\end{enumerate}

VSCode ima veliko funkcionalnosti, ki nam lahko pomagajo pri programiranju.
Več o tem v uradni dokumentaciji: \url{https://code.visualstudio.com/docs/editor/codebasics}

\hypertarget{numpy}{%
\section{Numpy}\label{numpy}}

\hypertarget{pogoste-teux17eave}{%
\section{Pogoste težave}\label{pogoste-teux17eave}}

\hypertarget{uvod-v-ptyhon}{%
\chapter{Uvod v Ptyhon}\label{uvod-v-ptyhon}}

Stran je v delu.

\hypertarget{zanke}{%
\chapter{Zanke}\label{zanke}}

Stran je v delu.

\hypertarget{seznami}{%
\chapter{Seznami}\label{seznami}}

Stran je v delu.

\hypertarget{delo-z-objekti}{%
\chapter{Delo z objekti}\label{delo-z-objekti}}

Gradivo za to poglavje je \url{https://automatetheboringstuff.com/2e/chapter6/}

Metod je veliko, spodaj je naštetih nekaj najpogosteje uporabljanih. Celoten
seznam je v uradni dokumentaciji: \url{https://docs.python.org/3/library/stdtypes.html\#string-methods}

Ponavadi lahko z Googlom, najdemo metodo, ki jo potrebujemo, če opišemo, kaj želimo
narediti (npr. s \texttt{python\ count\ characters\ in\ string} hitro najdemo \texttt{count()} in primere uporabe).

\hypertarget{metode-za-nize}{%
\section{Metode za nize}\label{metode-za-nize}}

\begin{itemize}
\tightlist
\item
  \texttt{niz.count(znak)} vrne kolikokrat se znak pojavi v nizu
\item
  \texttt{niz.index(znak)} vrne indeks, na katerem se znak prvič pojavi
\item
  \texttt{niz.replace(prviniz,\ druginiz)} vrne niz, kjer so vsi nizi enaki \texttt{prviniz}
  zamenjani z \texttt{druginiz}
\item
  \texttt{niz.lower()} in \texttt{niz.upper()} vrne niz, kjer iz malih črk naredi velike ali obratno
\item
  \texttt{niz.islower()} in \texttt{niz.isupper()}
\item
  \texttt{niz.strip()} vrne niz, kjer z leve in desne strani odstrani ``whitespace characters'' (presledki, tab, \texttt{"\textbackslash{}n"}). Lahko podamo neobvezni argument, s katerim določimo, katere znake
  naj odstrani z leve in desne. Obstajata tudi metodi \texttt{rstrip()} in \texttt{lstrip()}, ki odstranjujeta le z leve in desne.
\end{itemize}

\begin{Shaded}
\begin{Highlighting}[]
\BuiltInTok{print}\NormalTok{(}\StringTok{\textquotesingle{}    Hello, World    }\CharTok{\textbackslash{}n}\StringTok{\textquotesingle{}}\NormalTok{.strip())}
\end{Highlighting}
\end{Shaded}

\begin{verbatim}
## Hello, World
\end{verbatim}

\begin{itemize}
\tightlist
\item
  \texttt{"locilo".join(seznam)} združi elemente seznama v niz in postavilo \texttt{locilo}
  med posamezne elemente
\end{itemize}

\begin{Shaded}
\begin{Highlighting}[]
\BuiltInTok{print}\NormalTok{(}\StringTok{\textquotesingle{}ABC\textquotesingle{}}\NormalTok{.join([}\StringTok{\textquotesingle{}Moje\textquotesingle{}}\NormalTok{, }\StringTok{\textquotesingle{}ime\textquotesingle{}}\NormalTok{, }\StringTok{\textquotesingle{}je\textquotesingle{}}\NormalTok{, }\StringTok{\textquotesingle{}Rok\textquotesingle{}}\NormalTok{]))}
\end{Highlighting}
\end{Shaded}

\begin{verbatim}
## MojeABCimeABCjeABCRok
\end{verbatim}

\begin{itemize}
\tightlist
\item
  \texttt{niz.split(locilo)} vrne seznam, kjer so elementi posamezni deli niza, ki jih
  ločuje \texttt{locilo}. Privzeta vrednost za \texttt{locilo} je presledek.
\end{itemize}

\begin{Shaded}
\begin{Highlighting}[]
\BuiltInTok{print}\NormalTok{(}\StringTok{"Moje ime je Rok."}\NormalTok{.split())}
\end{Highlighting}
\end{Shaded}

\begin{verbatim}
## ['Moje', 'ime', 'je', 'Rok.']
\end{verbatim}

\hypertarget{numpy-1}{%
\chapter{Numpy}\label{numpy-1}}

Stran je v delu.

\hypertarget{datoteke}{%
\chapter{Datoteke}\label{datoteke}}

Gradivo za to poglavje je \url{https://automatetheboringstuff.com/2e/chapter9/}

\hypertarget{datoteux10dni-sistem}{%
\section{Datotečni sistem}\label{datoteux10dni-sistem}}

Datoteke so shranjene na različnih nosilcih (npr. trdi disk, SSD, DVD, USB ključ, \ldots).
Na računalniku datoteke organiziramo po mapah, ki so lahko gnezdene. Na vrhu
imamo korensko mapo (root folder). Na Linux in macOS je to \texttt{/} na Windowsu pa
\texttt{C:\textbackslash{}}, kjer je \texttt{C} ime particije.

Prostor, ki je na voljo na nosilcu lahko razdelimo
na več ločenih delov, ki jim rečemo particije. npr. trdi disk z
1000 GB bi lahko razdelili na dve particiji \texttt{C} z 100 GB in \texttt{D} z 900 GB). Vsaka
particija na nosilcih, ki so priklopljeni na računalnik, dobi svojo črko
(to velja za Windows, drugje je drugače). npr.
USB ključi so pogosto pod \texttt{E} ali \texttt{F}.

\hypertarget{absolutna-in-relativna-pot}{%
\subsection{Absolutna in relativna pot}\label{absolutna-in-relativna-pot}}

Vsaki datoteki ustreza ena absolutna pot. To je ``naslov'', pod katero jo lahko najdemo.
Primer: \texttt{C:\textbackslash{}eggs\textbackslash{}bacon\textbackslash{}spam.txt}. Pot vsebuje vse mape, v katerih se datoteka
nahaja, ločene z \texttt{\textbackslash{}} (na Windowsu; na Linux in macOS je ločilo \texttt{/}), ime datoteke,
piko in končnico datoteke, ki določa njen tip.

Relativna pot do datoteke je pot glede na neko drugo mapo. Za zgornji primer:
glede na mapo \texttt{eggs} je relativna pot do datoteke \texttt{.\textbackslash{}bacon\textbackslash{}spam.txt}, kjer pika
pomeni trenutno mapo. Če bi imeli mapo \texttt{tomatoes} v mapi \texttt{C:} in v njej datoteko \texttt{dat.txt},
bi bila relativna pot glede na \texttt{eggs} enaka \texttt{..\textbackslash{}tomatoes\textbackslash{}dat.txt}. Dve piki pomenita
eno mapo višje v hierarhiji (parent folder) glede na trenutno mapo.
Če bi želeli iti dve mapi višje bi uporabili \texttt{..\textbackslash{}..\textbackslash{}nekadrugamapa}, itd.

Za podrobnejši razlago in več primerov glej gradivo: \url{https://automatetheboringstuff.com/2e/chapter9/}

\hypertarget{delo-z-ukaznim-pozivom}{%
\subsection{Delo z ukaznim pozivom}\label{delo-z-ukaznim-pozivom}}

Podobno kot v Raziskovalcu (File Explorer) se tudi v ukaznem pozivu (Terminal) v
nekem trenutku nahajamo v neki mapi (ang. Current working directory ali CWD).
Ta mapa je vedno napisana na začetku vrstice.
V ukaznem pozivu najprej napišemo ukaz nato parametre, ki jih želimo podati, ločene
s presledki. Ukaz izvedemo s tipko Enter.

V neko mapo se lahko premaknemo z ukazom \texttt{cd}, ki mu kot argument podamo pot do
mape, v katero se želimo premakniti.

Ukaz \texttt{dir} izpiše vse datoteke in mape, ki se nahajajo v trenutni mapi.

Glej tudi: \url{https://ucilnica.fmf.uni-lj.si/mod/page/view.php?id=2505}

\hypertarget{mape-in-datoteke-v-pythonu}{%
\subsection{Mape in datoteke v Pythonu}\label{mape-in-datoteke-v-pythonu}}

Za delo z datotečnim sistemom je na voljo knjižnica \texttt{os}.
Posamezne funkcije in njihovo uporabo lahko poiščete v uradni dokumentaciji. Nekaj
najbolj uporabnih je \texttt{os.getcwd()}, \texttt{.chdir()}, \texttt{.listdir()}, \texttt{.mkdir()},
\texttt{.rename()}, \texttt{.remove()}, \texttt{.rmdir()}.

Za delo s potmi je uporabna knjižnica \texttt{os.path}, kjer so uporabne funkcije
\texttt{os.path.join()}, \texttt{.exists()}, \texttt{.abspath()}, \texttt{.relpath()}, \texttt{.isfile()}.

\hypertarget{pisanje}{%
\section{Pisanje}\label{pisanje}}

Datoteko odpremo v načinu za pisanje \texttt{mode="w"} in uporabimo funkcijo \texttt{write()},
ki zapiše niz v datoteko. Znak \texttt{\textbackslash{}n} pomeni novo vrstico. Če želimo zapisati znak \texttt{\textbackslash{}}
moramo v Pythonu napisati \texttt{\textbackslash{}\textbackslash{}}. Več o uporabi \texttt{\textbackslash{}} v Pythonu: \url{https://www.w3schools.com/python/gloss_python_escape_characters.asp}

\begin{Shaded}
\begin{Highlighting}[]
\NormalTok{potdodatoteke }\OperatorTok{=} \StringTok{"datoteka.txt"}
\ControlFlowTok{with} \BuiltInTok{open}\NormalTok{(potdodatoteke, mode}\OperatorTok{=}\StringTok{"w"}\NormalTok{, encoding}\OperatorTok{=}\StringTok{"utf{-}8"}\NormalTok{) }\ImportTok{as}\NormalTok{ dat:}
\NormalTok{    dat.write(}\StringTok{"To je "}\NormalTok{)}
\NormalTok{    dat.write(}\StringTok{"en stavek.}\CharTok{\textbackslash{}n}\StringTok{To je drugi."}\NormalTok{)}
\end{Highlighting}
\end{Shaded}

\begin{verbatim}
## datoteka.txt
## To je en stavek.
## To je drugi.
\end{verbatim}

Namesto \texttt{dat.write("niz")} se lahko uporablja tudi \texttt{print("niz",\ file=dat)}, kjer
odprto datoteko podamo kot parameter.

\hypertarget{branje}{%
\section{Branje}\label{branje}}

\hypertarget{read}{%
\subsection{read()}\label{read}}

Datoteko odpremo v načinu za branje \texttt{mode="r"} in uporabimo metodo \texttt{read()},
ki vrne celotno vsebino datoteke naenkrat v obliki niza.

\begin{Shaded}
\begin{Highlighting}[]
\ControlFlowTok{with} \BuiltInTok{open}\NormalTok{(}\StringTok{"datoteka.txt"}\NormalTok{, mode}\OperatorTok{=}\StringTok{"r"}\NormalTok{, encoding}\OperatorTok{=}\StringTok{"utf{-}8"}\NormalTok{) }\ImportTok{as}\NormalTok{ datoteka:}
\NormalTok{    vsebina }\OperatorTok{=}\NormalTok{ datoteka.read()}
\BuiltInTok{print}\NormalTok{(vsebina)}
\end{Highlighting}
\end{Shaded}

\begin{verbatim}
## To je en stavek.
## To je drugi.
\end{verbatim}

Uporaba argumenta \texttt{mode} je opisana na dnu strani.
Klicu \texttt{open} lahko podamo tudi neobvezni argument \texttt{encoding}, ki poda kodno
tabelo, v kateri je napisana datoteka. Privzeta vrednost tega argumenta je na
Windowsu \texttt{cp1250}, kar je nekoliko zastarel standard, zato je dobra praksa
uporaba parametra \texttt{encoding="utf-8"}, s čimer uporabimo Unicode, ki se danes
uporablja skoraj povsod. Na macOS in Linux je vrednost \texttt{utf-8} že privzeta.

\hypertarget{readlines}{%
\subsection{readlines()}\label{readlines}}

Z metodo \texttt{readlines()} dobimo seznam, v katerem so posamezne
vrstice iz datoteke.

\begin{Shaded}
\begin{Highlighting}[]
\ControlFlowTok{with} \BuiltInTok{open}\NormalTok{(}\StringTok{"datoteka.txt"}\NormalTok{, mode}\OperatorTok{=}\StringTok{"r"}\NormalTok{, encoding}\OperatorTok{=}\StringTok{"utf{-}8"}\NormalTok{) }\ImportTok{as}\NormalTok{ datoteka:}
\NormalTok{    vrstice }\OperatorTok{=}\NormalTok{ datoteka.readlines()}
\BuiltInTok{print}\NormalTok{(vrstice)}
\end{Highlighting}
\end{Shaded}

\begin{verbatim}
## ['To je en stavek.\n', 'To je drugi.']
\end{verbatim}

\hypertarget{zanka}{%
\subsection{zanka}\label{zanka}}

Po vrsticah datoteke lahko gremo z zanko for.

\begin{Shaded}
\begin{Highlighting}[]
\NormalTok{vrstice }\OperatorTok{=}\NormalTok{ []}
\ControlFlowTok{with} \BuiltInTok{open}\NormalTok{(}\StringTok{"datoteka.txt"}\NormalTok{, mode}\OperatorTok{=}\StringTok{"r"}\NormalTok{, encoding}\OperatorTok{=}\StringTok{"utf{-}8"}\NormalTok{) }\ImportTok{as}\NormalTok{ datoteka:}
    \ControlFlowTok{for}\NormalTok{ line }\KeywordTok{in}\NormalTok{ datoteka:}
\NormalTok{        vrstice.append(line)}
\BuiltInTok{print}\NormalTok{(vrstice)}
\end{Highlighting}
\end{Shaded}

\begin{verbatim}
## ['To je en stavek.\n', 'To je drugi.']
\end{verbatim}

\hypertarget{mode}{%
\section{Mode}\label{mode}}

je neobvezni argument funkcije \texttt{open()}. Privzeta vrednost je \texttt{mode="rt"}. Zato
nam v zgornjih primerih ni bilo treba pisati \texttt{t} (je že privzet poleg druge
črke, ki jo podamo (\texttt{r} ali \texttt{w})). S posameznimi črkami povemo, kaj želimo z
datoteko početi.

\begin{longtable}[]{@{}lll@{}}
\toprule
oznaka & opis & opomba \\
\midrule
\endhead
r & branje & če ne obstaja, javi napako \\
w & pisanje & če ne obstaja, ustvari novo, izbriše prejšnjo vsebino datoteke \\
a & append & če ne obstaja, ustvari novo, ne izbriše prejšnje vsebine \\
x & ustvari datoteko, pisanje & če že obstaja, javi napako \\
+ & pisanje in branje & \\
t & za delo s tekstovnimi datotekami & npr. .txt, .csv, .tex, .html, .py \\
b & za delo s binarnimi datotekami & npr. slike \\
\bottomrule
\end{longtable}

Nekaj lastnosti je zbranih v spodnji tabeli:

\begin{longtable}[]{@{}lcccccccc@{}}
\toprule
lastnost     \textbackslash{}     kombinacija črk & r & r+ & x & x+ & w & w+ & a & a+ \\
\midrule
\endhead
branje & x & x & & x & & x & & x \\
pisanje & & x & x & x & x & x & x & x \\
datoteka mora obstajati & x & x & & & & & & \\
datoteka ne sme obstajati & & & x & x & & & & \\
zbriše prejšnjo vsebino datoteke & & & & & x & x & & \\
pisanje na konec datoteke & & & & & & & x & x \\
\bottomrule
\end{longtable}

K zgornjim kombinacijam lahko dodamo še \texttt{t} ali \texttt{b}.

\end{document}
